\documentclass[12pt,a4paper]{scrartcl}
\usepackage[utf8]{inputenc}
\usepackage[ngerman]{babel}
\usepackage{amsmath}
\usepackage{amsfonts}
\usepackage{amssymb}
\usepackage{blindtext}
\usepackage[left=2.5cm,right=1.5cm,top=2cm,bottom=2.8cm]{geometry}


\usepackage{fancyhdr}
\pagestyle{fancy}

\begin{document}
\begin{titlepage}
\begin{center}

\vspace*{3cm}
\textbf{\huge{Projektarbeit}}\\
\vspace*{2cm}
\textbf{\large{Entwicklung eines 2D-Spiels mit SFML}}\\
\vspace*{5cm}
Gabriel Gavrilas, G3C\\
Jan Kunzmann, G3C\\
Patrick Eigensatz, G3C
\end{center}
\end{titlepage}




\newpage

\setcounter{page}{1}
\section*{Vorwort}
\blindtext[1]

\newpage


\tableofcontents

\newpage


\section{Motivation}
\subsection{Warum ein 2D-Spiel?}
Als wir uns für ein Projektthema entscheiden mussten, haben
wir uns die Entscheidung schwer gemacht. Aus 5 verschiedenen
Bereichen, für die wir uns alle sehr interessierten, entschieden wir uns für die
eines Computerspieles.

\subsection{Warum C++ und SFML?}
Um die Sache für uns attraktiv zu machen,
wählten wir bewusst eine Programmiersprache, die noch nicht
alle von uns beherrschten. Natürlich hätten wir genau so gut
SDL oder direkt das darunterliegende OpenGL verwenden können. OpenGL
schied aus, da der Aufwand bereits ein einfaches 2D-Spiel zu realisieren,
schlichtweg nicht möglich gewesen wäre. SDL war unser Favorit, bis wir
SFML entdeckten. Ganz im Gegensatz zu SDL schien SFML für C++ ausgerichtet
zu sein. So wurden Klassen anstatt Strukturen verwendet, was den (für den
Anfang komplizierten) Umgang mit Zeigern reduzierte. Ausserdem besitzen die
Klassen eigene Konstruktoren, bzw. Destruktoren, was das Initialisieren
oder das freigeben von Speicher überflüssig macht. Davon erhofften wir uns
weniger Speicherzugriffsfehler und ein schnelleres programmieren. SFML
überzeugte uns schlussendlich, als wir die in der offiziellen Dokumentation
gezeigten Beispielprogramme angeschaut haben.


\end{document}
